\documentclass[11pt]{article}
\usepackage{amssymb,latexsym,amsmath,amsthm,graphicx, cite}
\usepackage{hyperref}
\hypersetup{
     colorlinks=true,
     linkcolor=blue,
     filecolor=blue,
     citecolor = black,
     urlcolor=blue,
     }
%\usepackage[sc]{mathpazo}
%\linespread{1.05}         % Palatino needs more leading (space between lines)
%\usepackage[T1]{fontenc}
\usepackage{mathptmx}
\usepackage{multirow}
\usepackage{float}
\restylefloat{table}
\hoffset=0in
\voffset=-.3in
\oddsidemargin=0in
\evensidemargin=0in
\topmargin=0in
\textwidth=6.5in
\textheight=8.8in
\marginparwidth 0pt
\marginparsep 10pt
\headsep 10pt

\theoremstyle{definition}  % Heading is bold, text is roman
\newtheorem{theorem}{Theorem}
\newtheorem{corollary}{Corollary}
\newtheorem{definition}{Definition}
\newtheorem{example}{Example}
\newtheorem{proposition}{Proposition}

\author{Chris Camano: ccamano@sfsu.edu}
\title{MATH 370  Homework 4}
\date

\newcommand{\Z}{\mathbb{Z}}
\newcommand{\N}{\mathbb{N}}
\newcommand{\Q}{\mathbb{Q}}
\newcommand{\R}{\mathbb{R}}
\newcommand{\C}{\mathbb{C}}

\newcommand{\lcm}{\mathrm{lcm}}



\setlength{\parskip}{0cm}
%\renewcommand{\thesection}{\Alph{section}}
\renewcommand{\thesubsection}{\arabic{subsection}}
\renewcommand{\thesubsubsection}{\arabic{subsection}.\arabic{subsubsection}}
\bibliographystyle{amsplain}

%\input{../header}

\newcommand{\nlim}{\lim_{n\rightarrow \infty}}
\begin{document}
\maketitle
\begin{enumerate}
  \item 9.8 (no proofs needed)
  \begin{enumerate}
    \item
    \[
      \nlim n^3=\infty
    \]
    \item
    \[
      \nlim (-n^3)=-\infty
    \]
    \item
    \[
      \nlim (-n^n)=NOT EXIST
    \]
    \item
      \[
        \nlim (1.01)^n=\infty
      \]
    \item
    \[
      \nlim(n^n)=\infty
    \]
  \end{enumerate}
  %%%%%%%%%%%%%%%%%%%%%%%%%%%%%%%%%%%%%%%%%%%%%
  \item  9.17 (proof needed):\\
  Give a formal proof that
  \[
    \lim_{n\rightarrow \infty}(n^2)=+\infty
  \]
  using definition 9.8:
  \textbf{Definition 9.8}\\
  For a squence($s_n$) we write $$\nlim(s_n)=+\infty$$ provided: $$\forall_{M>0}\exists_{N}:n>N\rightarrow s_n>M$$
  \begin{proof}
    To prove this fact we must choose some value for N such that the sequence $s_n$ for any n greater than N $s_n$ is greater than M. To do this let us consider the relationship between M and N. To satisfy the requirement under the definition of the function, let N=$\sqrt{m}$ then we have the following for n greater than our choice:
    \begin{align*}
      &n>N\\
      &n>\sqrt{M}\\
      &n^2> M\\
      &s_n>M
    \end{align*}
    So we have proven that with a selection of $N=\sqrt{M}$ then $n>N$ implies $s_n>M$
  \end{proof}
  %%%%%%%%%%%%%%%%%%%%%%%%%%%%%%%%%%%%%%%%%%%%%
  \item 9.18
  \begin{proof}
    \begin{enumerate}
      %%%%%%%%%%%%%%%%%%%%%%%%%%%%%%%%%%%%%%%
      \item Verify:
      \[
        \sum_{k=0}^na^k=\frac{1-a^{n+1}}{1-a}, a\neq 1
      \]
      \begin{proof}
        Base case: n=1:
        \begin{align*}
          &a^0+a^1=\frac{1-a^2}{1-a}\\
          &1+a=\frac{1-a^2}{1-a}\\
          &1+a=\frac{(1+a)(1-a)}{1-a}\\
          &1+a=1+a
        \end{align*}
        $$
        P(k):\sum_{i=0}^ka^i=\frac{1-a^{k+1}}{1-a}, a\neq 1$$
        \text{P(k+1)}\\
        \begin{align*}
          &\sum_{i=0}^{k+1}a^i=\frac{1-a^{k+2}}{1-a}\\\\
          &\sum_{i=0}^{k}a^i+a^{k+1}=\frac{1-a^{k+2}}{1-a}\\\\
          &\frac{1-a^{k+1}}{1-a}+a^{k+1}=\frac{1-a^{k+2}}{1-a}\\\\
          &\frac{1-a^{k+1}+a^{k+1}(1-a)}{1-a}=\frac{1-a^{k+2}}{1-a}\\\\
          &\frac{1-a^{k+1}+a^{k+1}-a^{k+2}}{1-a}=\frac{1-a^{k+2}}{1-a}\\\
          &\frac{1-a^{k+2}}{1-a}=\frac{1-a^{k+2}}{1-a}
        \end{align*}
      \end{proof}
      %%%%%%%%%%%%%%%%%%%%%%%%%%%%%%%%%%%%%%%
      \item \textbf{Find $\nlim(\sum_{k=0}^na^k), |a|<1}$}\\
      Find \[
        \nlim \frac{1-a^{n+1}}{1-a}
      \]
      \[
        \nlim \frac{1}{1-a}1-a^{n+1}
      \]
      \begin{align*}
        &\nlim \left(\frac{1}{1-a}\right)\nlim(1-a^{n+1})\\
        &\nlim \left(\frac{1}{1-a}\right)\nlim(1)-\nlim(a^{n+1})\\
        &\nlim\frac{1}{1-a}=\frac{1}{1-a}
      \end{align*}
      %%%%%%%%%%%%%%%%%%%%%%%%%%%%%%%%%%%%%%%
      \item
      \textbf{Calculate $\nlim(\sum_{k=0}^n\frac{1}{3^k})$}
    \[
      \sum_{k=0}^n\frac{1}{3^k}=\sum_{k=0}^n\left(\frac{1}{3}\right)^k
    \]
    By part b we know that this sequence can be solved in the limit as:
    \[
      \frac{1}{1-a}
    \]
    thus:
    \[
      \nlim\sum_{k=0}^n\frac{1}{3^k}=\frac{1}{\frac{2}{3}}=\frac{3}{2}
    \]
      %%%%%%%%%%%%%%%%%%%%%%%%%%%%%%%%%%%%%%%
      \item
      \textbf{What is  $\nlim(\sum_{k=0}^na^k), a\geq 1}$}
      \begin{align*}
        &\nlim \left(\frac{1}{1-a}-\frac{a^{n+1}}{{1-a}}\right)\\
        &\left(\frac{1}{1-a}-\frac{\nlim a^{n+1}}{{1-a}}\right)\\
        &\left(\frac{1}{1-a}-\frac{\infty}{{1-a}}\right)=\infty\\
      \end{align*}
    \end{enumerate}
  \end{proof}
  %%%%%%%%%%%%%%%%%%%%%%%%%%%%%%%%%%%%%%%%%%%%%
  \item 10.1
  \begin{proof}
    Which of the sequences are increasing? decreasing? Bounded?
    \begin{enumerate}
      \item
      \[
        s_n=\frac{1}{n}
      \]
      The sequence above converges to zero, and thus by theorem 9.1 is bounded. The sequence is decreasing as $s_n \geq s_{n+1} \forall n$
      \item
      \[
        s_n=\frac{(-1^n)}{n^2}
      \]
      This sequence is bounded but does not converge and is not monotonic
      \item
      \[
        s_n=n^5
      \]
      This sequence is increasing, is not bounded and does not converge
      \item
      \[
        s_n=sin(\frac{n\pi}{7})
      \]
      This sequence is bounded but does not converge and is not monotonic
      \item
      \[
        s_n=(-2)^n
      \]
      This sequence is not bounded, and is not monotonic
      \item
      \[
        s_n=\frac{n}{3^n}
      \]
      This sequence is decreasing and is bounded it converges as well.
    \end{enumerate}
  \end{proof}
  %%%%%%%%%%%%%%%%%%%%%%%%%%%%%%%%%%%%%%%%%%%%%
  \item 10.6
  \begin{proof}
    \item\\
    Let $(s_n)$ be a sequence such that:
    \[
      |s_{n+1}-s_n|<2^{-n} \quad \forall n\in \mathbb{N}
    \]
    Prove $s_n$ is a Cauchy sequence and hence a convergent sequence.
    \begin{proof}
    To prove that $s_n$ is a Cauchy sequence let us prove then that \[
      \forall \epsilon>0 \exists N: \quad m,n>N \righatrrow |s_m-s_n|<\epsilon
    \]
    In order to go about proving this statement we must first find a way to connect the original statement to a general statment about being cauchy. One such way of doing this is through estbalishing the following corresponce:
    \[
      m=n+k, k \in \Z
    \]
    such that:
    \[
      |s_m-s_n|<\epsilon\rightarrow |s_{n+k}-s_n|<\epsilon
      \]
    \end{proof}
    With this new construction we can now leverage the triangle inequality in the following way for each pairwise distance estimate:
    \[
      |s_{n+k}-s_n|\leq \sum_{i=n}^{n+k}|s_{i+1}-s_i|
    \]
    however, note also that we have the original inequality for the case of k=1, meaning we can develop this inequality further:
    \[
      |s_{n+k}-s_n|\leq \sum_{i=n}^{n+k}|s_{i+1}-s_i|<\sum_{i=n}^{n+k-1}\frac{1}{2^i}
    \]
    which simplifies to:
    \[
      |s_{n+k}-s_n|<\sum_{i=n}^{n+k-1}\frac{1}{2^i}
    \]
    We would like to use the geometric series formula here to simplify the right hand side however the starting index is problematic, so to correct this we can factor out a term of: $\frac{1}{2^k}$ leaving:
    \[
      |s_{n+k}-s_n|<\frac{1}{2^n}\sum_{i=0}^{k-1}\frac{1}{2^i}
    \]
    Applying formula for sum of a geometric sequence:
    \[
        |s_{n+k}-s_n|<\frac{1}{2^n}\left[\frac{1-\frac{1}{2^k}}{\frac{1}{2}}\right]<\frac{1}{2^n}\left[\frac{1}{\frac{1}{2}}\right]
    \]
    \[
      |s_{n+k}-s_n|<\frac{1}{2^{n-1}}
    \]
    From this statement we can now prove that it is cauchy by considering an epsilon as follos:
    \[
      \epsilon=\frac{1}{2^{n-1}}
    \]
    Which gives a value of N equal to:
    \[
      N=\log_2(\frac{1}{\epsilon})+1
    \]
    Finally leeaving us to the statement:
    $$\forall \epsilon >0 \exists N=\log_2(\frac{1}{\epsilon})+1: n+k,n>N \rightarrow |s_{n+k}-s_n|<\epsilon$$
    Thus since $s_n$ is cauchy by the completeness of the real numbers $s_n$ is also convergent.\\\\
    \item\\
    Is the result in (a) true if we only assume $|s_{n+1}-s_n|<\frac{1}{n}$?
    \\\\ If we make this assumption then we cannot say that the sequence is cauchy for the following reason:
    When we start to consider the same set up and get to a sum over $\frac{1}{n}$ we are now working with the harmonic series, or rather a partial sum of said harmonic series. This sequence diverges to infinity(slowly) Meaning we can always pick values greater than some $N$ such that the distance between terms is not fixed in the desired cauchy manner for $|s_{n+k}-s_n|$.
  \end{proof}
  %%%%%%%%%%%%%%%%%%%%%%%%%%%%%%%%%%%%%%%%%%%%%
  \item State Boltzano-Weierstrass Theorem. Apply it to $a_n$= cos(2 pi n) and illustrate an example of a subsequence.\\
  The Boltzano-Weierstrass Theorem is as follows:
  \textbf{Every bounded sequence has a convergent subsequence}\\
  Consider the subsequence of $a_n=cos(2\pi n)$ in which we only consider the values $k\in \mathbb{N}$ then we have the subsequence $\{1,1,1,1,1,\}$ which has $\
  lim_{k\rightarrow \infty}a_n_k=1$
  %%%%%%%%%%%%%%%%%%%%%%%%%%%%%%%%%%%%%%%%%%%%%
  \item 11.1: Let $a_n=3+2(-1)^n$
  State the first five terms of the sequence:
  \[
    a_n=\{1,5,1,5,1,5,1,5,1,5,....\}
  \]
  Give a subsequence that is constant with selection function: $\sigma$. One example of a subsequence that is constant is the choice of either the odd values or the even values each corresponding with one of the two possible values.
  \[
    s_{nk}=\{1,1,1,1,1,....,\} \quad \sigma(k)=2k+1
  \]
  Here we are picking the subsequence of the odd values of the sequence giving all ones since every other term is 1.
  %%%%%%%%%%%%%%%%%%%%%%%%%%%%%%%%%%%%%%%%%%%%%
  \item 11.2  Consider the following sequences:
  \[
    a_n=(-1)^n\quad b_n=\frac{1}{n} \quad c_n=n^2\quad d_n=\frac{6n+4}{7n-3}
  \]
  \begin{enumerate}
    \item For each sequence give an example of a monotone subsequence
    \begin{align*}
      &a_n: a_{nk}=\{-1,-1,-1,-1..\} \sigma(k)=2k+1
    \end{align*}
    All other sequences are monotone so any of their subsequences should be monotone as well ( this could get me in trouble but intuitivley makes sense).
    \item For each sequence give its set of subsequential limits
    By theorem 11.3 we have that if a sequence converges then every subsequence converges to the same limit. ( cool theorem ) Three of the sequence above all but the first have a limit thus their subsequence limits are equal to the sequence limits as follows:
    \[
      \nlim b_n=0\quad \nlim c_n=\infty\nlim d_n=\frac{6}{7}
    \]
    So for these sequences the subsequence limits are single elements sets consisting of just these values. For $a_n$ however we see some special behavior due to the fact that there is an alternating behavior in the sequence between -1 and 1 so the subsequential limits is the set $\{-1,1\}$
    \item For each sequence give its limsup and lim inf
    \[
      \liminf a_n=-1 \quad \limsup a_n=1
    \]
    By theorem 10.7 we have $\limsup=\liminf=\lim a_n$
    \[
      \liminf b_n=\lim b_n=0 \quad \limsup b_n=\lim a_b=0
    \]
    \[
      \liminf c_n=\lim c_n=\infty \quad \limsup c_n=\lim c_n=\infty
    \]
    \[
      \liminf d_n=\lim d_n=\frac{6}{7} \quad \limsup d_n=\lim d_n=\frac{6}{7}
    \]

    \item Which of the sequences converges? Diverges to $\infty$? Diverges to $-\infty$\\
    An does not converge, $b_n$ converges, $c_n$ diverges, $d_n$ converges
    \item Which of the sequences is bounded ?\\
    all sequences except $c_n$ are boundedjnk,
  \end{enumerate}
\end{enumerate}
\end{document}
