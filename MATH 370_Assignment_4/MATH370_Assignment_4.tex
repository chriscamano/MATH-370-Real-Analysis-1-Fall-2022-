\documentclass[11pt]{article}
\usepackage{amssymb,latexsym,amsmath,amsthm,graphicx, cite}
\usepackage{hyperref}
\hypersetup{
     colorlinks=true,
     linkcolor=blue,
     filecolor=blue,
     citecolor = black,
     urlcolor=blue,
     }
%\usepackage[sc]{mathpazo}
%\linespread{1.05}         % Palatino needs more leading (space between lines)
%\usepackage[T1]{fontenc}
\usepackage{mathptmx}
\usepackage{multirow}
\usepackage{float}
\restylefloat{table}
\hoffset=0in
\voffset=-.3in
\oddsidemargin=0in
\evensidemargin=0in
\topmargin=0in
\textwidth=6.5in
\textheight=8.8in
\marginparwidth 0pt
\marginparsep 10pt
\headsep 10pt

\theoremstyle{definition}  % Heading is bold, text is roman
\newtheorem{theorem}{Theorem}
\newtheorem{corollary}{Corollary}
\newtheorem{definition}{Definition}
\newtheorem{example}{Example}
\newtheorem{proposition}{Proposition}

\author{Chris Camano: ccamano@sfsu.edu}
\title{MATH 370  Homework 4}
\date

\newcommand{\Z}{\mathbb{Z}}
\newcommand{\N}{\mathbb{N}}
\newcommand{\Q}{\mathbb{Q}}
\newcommand{\R}{\mathbb{R}}
\newcommand{\C}{\mathbb{C}}

\newcommand{\lcm}{\mathrm{lcm}}



\setlength{\parskip}{0cm}
%\renewcommand{\thesection}{\Alph{section}}
\renewcommand{\thesubsection}{\arabic{subsection}}
\renewcommand{\thesubsubsection}{\arabic{subsection}.\arabic{subsubsection}}
\bibliographystyle{amsplain}

%\input{../header}

\newcommand{\nlim}{\lim_{n\rightarrow \infty}}
\begin{document}
\maketitle
\begin{enumerate}
  \item 9.8 (no proofs needed)
  \begin{enumerate}
    \item
    \[
      \nlim n^3=\infty
    \]
    \item
    \[
      \nlim (-n^3)=-\infty
    \]
    \item
    \[
      \nlim (-n^n)=NOT EXIST
    \]
    \item
      \[
        \nlim (1.01)^n=\infty
      \]
    \item
    \[
      \nlim(n^n)=\infty
    \]
  \end{enumerate}
  %%%%%%%%%%%%%%%%%%%%%%%%%%%%%%%%%%%%%%%%%%%%%
  \item  9.17 (proof needed):\\
  Give a formal proof that
  \[
    \lim_{n\rightarrow \infty}(n^2)=+\infty
  \]
  using definition 9.8:
  \textbf{Definition 9.8}\\
  For a squence($s_n$) we write $$\nlim(s_n)=+\infty$$ provided: $$\forall_{M>0}\exists_{N}:n>N\rightarrow s_n>M$$
  \begin{proof}
    To prove this fact we must choose some value for N such that the sequence $s_n$ for any n greater than N $s_n$ is greater than M. To do this let us consider the relationship between M and N. To satisfy the requirement under the definition of the function, let N=$\sqrt{m}$ then we have the following for n greater than our choice:
    \begin{align*}
      &n>N\\
      &n>\sqrt{M}\\
      &n^2> M\\
      &s_n<M
    \end{align*}
    So we have proven that with a selection of $N=\sqrt{M}$ then $n>N$ implies $s_n>M$
  \end{proof}
  %%%%%%%%%%%%%%%%%%%%%%%%%%%%%%%%%%%%%%%%%%%%%
  \item 9.18
  \begin{proof}
    \begin{enumerate}
      %%%%%%%%%%%%%%%%%%%%%%%%%%%%%%%%%%%%%%%
      \item Verify:
      \[
        \sum_{k=0}^na^k=\frac{1-a^{n+1}}{1-a}, a\neq 1
      \]
      \begin{proof}
        Base case: n=1:
        \begin{align*}
          &a^0+a^1=\frac{1-a^2}{1-a}\\
          &1+a=\frac{1-a^2}{1-a}\\
          &1+a=\frac{(1+a)(1-a)}{1-a}\\
          &1+a=1+a
        \end{align*}
        $$
        P(k):\sum_{i=0}^ka^i=\frac{1-a^{k+1}}{1-a}, a\neq 1$$
        \text{P(k+1)}\\
        \begin{align*}
          &\sum_{i=0}^{k+1}a^i=\frac{1-a^{k+2}}{1-a}\\\\
          &\sum_{i=0}^{k}a^i+a^{k+1}=\frac{1-a^{k+2}}{1-a}\\\\
          &\frac{1-a^{k+1}}{1-a}+a^{k+1}=\frac{1-a^{k+2}}{1-a}\\\\
          &\frac{1-a^{k+1}+a^{k+1}(1-a)}{1-a}=\frac{1-a^{k+2}}{1-a}\\\\
          &\frac{1-a^{k+1}+a^{k+1}-a^{k+2}}{1-a}=\frac{1-a^{k+2}}{1-a}\\\
          &\frac{1-a^{k+2}}{1-a}=\frac{1-a^{k+2}}{1-a}
        \end{align*}
      \end{proof}
      %%%%%%%%%%%%%%%%%%%%%%%%%%%%%%%%%%%%%%%
      \item \textbf{Find $\nlim(\sum_{k=0}^na^k), |a|<1}$}\\
      Find \[
        \nlim \frac{1-a^{n+1}}{1-a}
      \]
      \[
        \nlim \frac{1}{1-a}1-a^{n+1}
      \]
      \begin{align*}
        &\nlim \left(\frac{1}{1-a}\right)\nlim(1-a^{n+1})\\
        &\nlim \left(\frac{1}{1-a}\right)\nlim(1)-\nlim(a^{n+1})\\
        &\nlim\frac{1}{1-a}=\frac{1}{1-a}
      \end{align*}
      %%%%%%%%%%%%%%%%%%%%%%%%%%%%%%%%%%%%%%%
      \item
      \textbf{Calculate $\nlim(\sum_{k=0}^n\frac{1}{3^k})$}
    \[
      \sum_{k=0}^n\frac{1}{3^k}=\sum_{k=0}^n\frac{1}{3}^k
    \]
    By part b we know that this sequence can be solved in the limit as:
    \[
      \frac{1}{1-a}
    \]
    thus:
    \[
      \nlim\sum_{k=0}^n\frac{1}{3^k}=\frac{1}{\frac{2}{3}}=\frac{3}{2}
    \]
      %%%%%%%%%%%%%%%%%%%%%%%%%%%%%%%%%%%%%%%
      \item
      \textbf{What is  $\nlim(\sum_{k=0}^na^k), a\geq 1}$}
      \begin{align*}
        &\nlim \left(\frac{1}{1-a}-\frac{1-a^{n+1}}{{1-a}}\right)\\
        &\left(\frac{1}{1-a}-\frac{\nlim1-a^{n+1}}{{1-a}}\right)\\
        &\left(\frac{1}{1-a}-\frac{\infty}{{1-a}}\right)=\infty\\
      \end{align*}
    \end{enumerate}
  \end{proof}
  %%%%%%%%%%%%%%%%%%%%%%%%%%%%%%%%%%%%%%%%%%%%%
  \item 10.1
  \begin{proof}
    Which of the sequences are increasing? decreasing? Bounded?
    \begin{enumerate}
      \item
      \[
        s_n=\frac{1}{n}
      \]
      The sequence above converges to zero, and thus by theorem 9.1 is bounded. The sequence is decreasing as $s_n \geq s_{n+1} \forall n$
      \item
      \[
        s_n=\frac{(-1^n)}{n^2}
      \]
      This sequence is bounded but does not converge and is not monotonic
      \item
      \[
        s_n=n^5
      \]
      This sequence is increasing, is not bounded and does not converge
      \item
      \[
        s_n=sin(\frac{n\pi}{7})
      \]
      This sequence is bounded but does not converge and is not monotonic
      \item
      \[
        s_n=(-2)^n
      \]
      This sequence is not bounded, and is not monotonic
      \item
      \[
        s_n=\frac{n}{3^n}
      \]
      This sequence is decreasing and is bounded it converges as well.
    \end{enumerate}
  \end{proof}
  %%%%%%%%%%%%%%%%%%%%%%%%%%%%%%%%%%%%%%%%%%%%%
  \item 10.6
  \begin{proof}
    \item\\
    Let $(s_n)$ be a sequence such that:
    \[
      |s_{n+1}-s_n|<2^{-n} \quad \forall n\in \mathbb{N}
    \]
    Prove $s_n$ is a Cauchy sequence and hence a convergent sequence.
    \begin{proof}
    To prove that $s_n$ is a Cauchy sequence let us prove then that \[
      \forall \epsilon>0 \exists N: \quad m,n>N \righatrrow |s_n-s_m|<\epsilon
    \]
    \end{proof}
    \item\\
    Is the result in (a) true if we only assume $|s_{n+1}-s_n|<\frac{1}{n}$?
  \end{proof}
  %%%%%%%%%%%%%%%%%%%%%%%%%%%%%%%%%%%%%%%%%%%%%
\end{enumerate}
\end{document}
