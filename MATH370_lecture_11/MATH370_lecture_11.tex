\documentclass[11pt]{article}
\usepackage{amssymb,latexsym,amsmath,amsthm,graphicx, cite}
 \author{Chris Camano: ccamano@sfsu.edu}
 \title{MATH 370  lecture 11 }
 \date

\usepackage{mathptmx}
\usepackage{multirow}
\usepackage{float}
\restylefloat{table}
\hoffset=0in
\voffset=-.3in
\oddsidemargin=0in
\evensidemargin=0in
\topmargin=0in
\textwidth=6.5in
\textheight=8.8in
\marginparwidth 0pt
\marginparsep 10pt
\headsep 10pt

\theoremstyle{definition}  % Heading is bold, text is roman
\newtheorem{theorem}{Theorem}
\newtheorem{corollary}{Corollary}
\newtheorem{definition}{Definition}
\newtheorem{example}{Example}
\newtheorem{proposition}{Proposition}



\newcommand{\Z}{\mathbb{Z}}
\newcommand{\N}{\mathbb{N}}
\newcommand{\Q}{\mathbb{Q}}
\newcommand{\R}{\mathbb{R}}
\newcommand{\C}{\mathbb{C}}

\newcommand{\lcm}{\mathrm{lcm}}



\setlength{\parskip}{0cm}
%\renewcommand{\thesection}{\Alph{section}}
\renewcommand{\thesubsection}{\arabic{subsection}}
\renewcommand{\thesubsubsection}{\arabic{subsection}.\ar  abic{subsubsection}}
\bibliographystyle{amsplain}

%\input{../header}


\begin{document}
\maketitle

\sect{Opening notes and midterm conversation}\\

Sequences are functions from : $f:\N\mapsto \R$ Generally subsequences can be understood as a composition of two functions.
\sect{Proof of Bolzano Weirstrauss Theorem}\\
 any bounded sequence has a convergent monotonic  subsequence . \\\\
 \textbf{Lemma} Any sequence has a monotonic subsequence : Proof:
 Let $a_k$ is a dominant element, if for all $m>k a_m\leq a_k$  \\
 Case 1 : There are infinitley many dominant elements of $a_n$ The subsequence of dominant elements is monotonic3 and decreasing.
 \\
 Case 2: There are finitley many dominant elements. le m be the index such that $a_n$ is not dominant if $n>M$


\end{document}
