\documentclass[12pt]{article}
\usepackage[pdftex]{graphicx}
\usepackage{amsmath,amssymb,amsthm}
\usepackage{hyperref}
\pagestyle{empty}
\author{Chris Camano: ccamano@sfsu.edu}
\title{MATH 370  Homework 1 }
\date

\topmargin -0.6in
\headsep 0.40in
\oddsidemargin 0.0in
\textheight 9.0in
\textwidth 6.5in
\vfuzz2pt
\hfuzz2pt

%%%%Short cuts and formatting%%%%%%%%%%
\newcommand{\q}{\quad}
\newcommand{\tab}{\\\\}
\renewcommand{\labelenumi}{\alph{enumi})}
\newcommand{\sect}[1]{\section*{#1}}

%%%%%%Vector Spaces%%%%%%%%%%%%%%%%%%%
\newcommand{\R}{\mathbb{R}}
\newcommand{\C}{\mathbb{C}}
\newcommand{\F}{\mathbb{F}}
\newcommand{\Q}{\mathbb{Q}}
\newcommand{\rtwo}{\mathbb{R}^2}
\newcommand{\mxn}{{m\times n}}

%%%%%%Sets and common phrases%%%%%%%%%
\newcommand{\Axb}{\textbf{Ax=b} }
\newcommand{\Axz}{\textbf{Ax=0} }
\newcommand{\dim}{\text{dim}}
\newcommand{\lc}{linear combination }
\newcommand{\let}{\text{Let }}
\newcommand{\tf}{\therefore}
%%%%%%%%%Analysis%%%%%%%%%%%%%%%%%%%%%
\newcommand{\arr}{\rightarrow}
\newcommand{\xlim}{\lim_{x\rightarrow \infty}}
\newcommand{\Z}{\mathbb{Z}}
\newcommand{\N}{\mathbb{N}}
\newcommand{\ep}{\varepsilon}
\newcommand{\i}{\text{ if }}
\newcommand{\and}{\text{ and }}
%%%%%% Theorem formatting%%%%%%%%%%%
\newtheorem{thm}{Theorem}[section]
\newtheorem{cor}[thm]{Corollary}
\newtheorem{lem}[thm]{Lemma}
\newtheorem{prop}[thm]{Proposition}
\theoremstyle{definition}
\newtheorem{defn}[thm]{Definition}
\theoremstyle{remark}
\newtheorem{rem}[thm]{Remark}
\numberwithin{equation}{section}
\everymath={\displaystyle}


\begin{document}
\maketitle
For every rational number there exists a natural number such that their product is also a natural number
\[
  \forall\quad  p\in \Q \quad \exists  k \in \N : pk\in \N
\]
For every irrational number there does not exist a natural number such that there product is rational.
\[
  \forall\quad  a \in \mathbb{Q}^c \quad \nexists k\in \N: ak\in \N
\]\\\\
Prove the following:
\[
  n^4< 3^n
\]
For sufficiently large x :
\[
  \frac{3^x}{x^n}
\]
\[
  \frac{3^x}{x^4}>1
\]
or
\[
  x^4<3^x
\]
This statement is true for all n greater than 8.
Suppose that
P(k):
\[
  n^4<3^n
\]
for $n\geq 8$.\\\\
P(k+1):
\[
  (n+1)^4<3^{n+1}
\]
\begin{align*}
  &k^4<3^k\\
  &3k^4<3^{k+1}\\
  &\frac{n+1}{n}^4\leq 3\\
  &\frac{n+1}{n}\leq 3^{\frac{1}{4}}\\
  &1+\frac{1}{n}\leq 3^{\frac{1}{4}}\\
  &n\geq \frac{1}{3^{\frac{1}{4}}-1}
\end{align*}
\defn Supremum and completeness over $\mathbb{R}$\\\\
Consider the following set:
\[
S=\{x\in \Q: x^2<2\}
\]
A property of the rational numbers is that for all subsets of the rationals there exists a smallest element. an upper bound can be thought of as the maximum value for a subset. The set of all upper bounds has a minimum denoted as the supermum of the set. \\\\
For all intervals over the real numbers there exists a supremum and infimum such that for all elements in a a is less than the supremum and all elements are greater than or equal that the infimum with the inequality equivilancy depending on if the set is open or closed. \\\\
Every bounded set has a least upper bound. \\\\
\defn Completness: Every bounded subset of R has a least upperbound. denoted as supremum of A. \\\\
Sup$\{x<100\}$=100
\end{document}
