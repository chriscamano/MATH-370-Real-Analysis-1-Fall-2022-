\documentclass[11pt]{article}
\usepackage{amssymb,latexsym,amsmath,amsthm,graphicx, cite}
 \author{Chris Camano: ccamano@sfsu.edu}
 \title{MATH 370  lecture 6 }
 \date

\usepackage{mathptmx}
\usepackage{multirow}
\usepackage{float}
\restylefloat{table}
\hoffset=0in
\voffset=-.3in
\oddsidemargin=0in
\evensidemargin=0in
\topmargin=0in
\textwidth=6.5in
\textheight=8.8in
\marginparwidth 0pt
\marginparsep 10pt
\headsep 10pt

\theoremstyle{definition}  % Heading is bold, text is roman
\newtheorem{theorem}{Theorem}
\newtheorem{corollary}{Corollary}
\newtheorem{definition}{Definition}
\newtheorem{example}{Example}
\newtheorem{proposition}{Proposition}



\newcommand{\Z}{\mathbb{Z}}
\newcommand{\N}{\mathbb{N}}
\newcommand{\Q}{\mathbb{Q}}
\newcommand{\R}{\mathbb{R}}
\newcommand{\C}{\mathbb{C}}

\newcommand{\lcm}{\mathrm{lcm}}



\setlength{\parskip}{0cm}
%\renewcommand{\thesection}{\Alph{section}}
\renewcommand{\thesubsection}{\arabic{subsection}}
\renewcommand{\thesubsubsection}{\arabic{subsection}.\arabic{subsubsection}}
\bibliographystyle{amsplain}

%\input{../header}


\begin{document}

\maketitle
\[
  lim_{n\mapsto \infty}a_n=L
\]
\begin{enumerate}
  \item outside every interval centered at L there is a finite number of $a_n's$
  \item Inside dvery interval containing L therae are infinite number of $a_n's$.
\end{enumerate}
\[
  lim_{n\mapsto \infty}a_n =\infty
\]
\begin{enumerate}
  \item Every bounded interval contains finite number of sequences
  \item outside every bonded interval there  are infinite sequences.
\end{enumerate}
Midterm structure: in person\\
Example of a question that may appear on the midterm. \\\\
Probably about three weeks from now.
\\\\
\newcommand{\nlim}{\lim_{n\rightarrow\infty}}
If  $\nlim a_n=L_1$ and $\nlim a_n=L_2$ then $L_1=L_2$\\\\
\defn $\nlim(ka_nn)=k\nlim a_n\quad k\in \Z$\\\\
Consider the following:
\newcommand{\eps}{\epsilon}
\[
   \forall \mu >0 \quad\exists N : \forall n>N : |a_n-L|<\mu
\]
Given an arbitray $\eps >0 $ let $\mu =\frac{\eps}{10}$
So this gives that:
\[
  \forall \eps >0\quad  \exists N : \forall n>N |a_n-L|<\frac{\eps}{10}
\]
\\\\
Proof of  $\nlim(ka_nn)=k\nlim a_n\quad k\in \Z$\\\\
\begin{proof}
  \[
    \forall \eps >0 \quad\exists N : \forall n>N |a_n-L|<\frac{\eps}{k}
  \]
\begin{align*}
  &k|a_n-L|<\eps\\
  &\forall \eps >0\quad  \exists N  |ka_n-kL|<\eps\\
  &\nlim (ka_n)=kL\\
  &\nlim (ka_n)=k\nlim a_n
\end{align*}
\end{proof}
Lemma: If there exists a finite limit then the sequence must be bounded. In the situation where we have a finite number of points hat exist outside of the limit strip we can add the maximum outlier to the limit to correct the inequality. \\
\begin{align*}
  &|a_n-L|<10\\
  &|a_n|<10+L\\
\end{align*}
Let M $=\max(|a_1|,|a_2|,...,|a_{n-1}|,10+L)$ then for all $a_n $ $a_n<M $ thus if the limit exists the the set is bounded.\\\\
Theorem: The limit of the prodcut of two sequences is equal to the limit of both as folows:
\[
  \nlim a_nb_n=\nlim a_n\nlim b_n
\]
Additional Theorems:
\[
  \nlim b_n \neq 0 \rightarrow \nlim \frac{a_n}{b_n}=\frac{\nlim a_n}{\nlim b_n}
\]
Additional Theorem:
\[
  \nlim \left(\frac{1}{n^p}\right)=0 \text{ if}  p>0
\]
Additional Theorem:
\[
  \nlim a^n=0 \text{ if } |a|<1
\]
Additional Theorem:
\[
  \nlim n^{\frac{1}{n}}=1
\]
Additional Theorem:
\[
  \nlim a^{\frac{1}{n}}=1, q\neq 0
\]

\end{document}
