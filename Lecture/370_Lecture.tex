\documentclass[12pt]{article}
\usepackage[pdftex]{graphicx}
\usepackage{amsmath}
\usepackage{amssymb}
\pagestyle{empty}
\author{Chris Camano: ccamano@sfsu.edu}
\title{MATH 370  Lecture 12 }
\date{8/23/22}

\topmargin -0.6in
\headsep 0.40in
\oddsidemargin 0.0in
\textheight 9.0in
\textwidth 6.5in

\newcommand{\econst}{\mathrm{e}}
\newcommand{\diff}{\mathrm{d}}
\newcommand{\dwrt}[1]{\frac{\diff}{\diff #1}}
%%%%%%Macros for 425%%%%%%%%%%%%%%%%%%%
\newcommand{\q}{\quad}
\newcommand{\tab}{\\\\}
\renewcommand{\labelenumi}{\alph{enumi})}
\newcommand{\sect}[1]{\section*{#1}}

%%%%%%Vector Spaces%%%%%%%%%%%%%%%%%%%
\newcommand{\R}{\mathbb{R}}
\newcommand{\C}{\mathbb{C}}
\newcommand{\F}{\mathbb{F}}
\newcommand{\rtwo}{\mathbb{R}^2}
\newcommand{\mxn}{{mxn}}

%%%%%%Sets and common phrases%%%%%%%%%
\newcommand{\Axb}{\textbf{Ax=b} }
\newcommand{\Axz}{\textbf{Ax=0} }
\newcommand{\dim}{\text{dim}}
\newcommand{\lc}{linear combination }
\newcommand{\let}{\text{Let }}
\newcommand{\tf}{\therefore}
%%%%%%%%%Analysis%%%%%%%%%%%%%%%%%%%%%
\newcommand{\arr}{\rightarrow}
\newcommand{\xlim}{\lim_{x\rightarrow \infty}}
\newcommand{\Z}{\mathbb{Z}}
\newcommand{\ep}{\epsilon}



\everymath={\displaystyle}


\begin{document}
\maketitle
\sect{First day of Lecture}
Office hours conducted in Thornton 932.\\
\textbf{Basic number systems}:\\\\
\textbf{ Integers }
\[
  \Z=\{..,-1,0,1,..\}
\]
\\\\
\textbf{ Natural Numbers }
\[
  \mathbb{N} =\{ 1,2,3,..\}
\]
\\\\
\textbf{Equivilance Classes}:
An equivilance class is a way of differentiating similar mathematical items such as rational numbers.\\\\
\textbf{ Rational Numbers }
\[
  \mathbb{Q} =\{ \frac{1}{1},-\frac{1}{1},-\frac{2}{1},\frac{2}{1},\frac{1}{2},\frac{1}{3},\frac{2}{2},\frac{3}{1},...\}
\]
\\\
The set of rational numbers is the set of numbers defined as the ratio of two integers.\\
Suppose that a real number line is defined over some finite set of integers. To guarentee that there exists some rational number within $[-\epsilon, \epsilon]$ I start by picking a number that I can guarentee is smaller than epsilon such as $10^k$.
\\
A rational number is sometimes referred to as a dent on the real number system. \\
\\
 \textbf{List}: A list is a funciton whose domain is the set of natural numbers \mathbb{N}.\\\\
Consider the following situation: Given p,q $\in \Z$ $$ \nexists\quad  p,q \in \Z : \left \frac{p}{q}\right^2=2$$
\textbf{Proof}:
Suppose \begin{align*}
    &\exists\quad  p,q \in \mathbb{N} :\\
    &(\frac{p}{q})^2=2\\
    &\therefore p^2=2q^2. \\
    &2|p \therefore p=2k, k \in \Z\\
    &2k^2=q^2\\
    &\therefore 2|q
\end{align*}
Since we have shown that q and p share a common divisor of 2, however this contradicts the notion that the rational number was expressed in reduced form at the start of the proof. This implies that $\sqrt{2} \notin \mathbb{N}$\\\\
\textbf{Dedekind Cuts} Dedekind cuts are used to express irrational numbers and culminate to the construction of the real numbers.\\\\
Eventually you start loosing algebraic properties as you expand number systems such as commutativity and associativity. \\\\
\sect{Limits}
\textbf{Limit}
\[
  \xlim f(x)=L
\]

\end{document}
