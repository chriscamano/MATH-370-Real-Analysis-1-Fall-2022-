\documentclass[11pt]{article}
\usepackage{amssymb,latexsym,amsmath,amsthm,graphicx, cite}
\usepackage{hyperref}
\hypersetup{
     colorlinks=true,
     linkcolor=blue,
     filecolor=blue,
     citecolor = black,
     urlcolor=blue,
     }

\usepackage{mathptmx}
\usepackage{multirow}
\usepackage{float}
\restylefloat{table}
\hoffset=0in
\voffset=-.3in
\oddsidemargin=0in
\evensidemargin=0in
\topmargin=0in
\textwidth=6.5in
\textheight=8.8in
\marginparwidth 0pt
\marginparsep 10pt
\headsep 10pt

\theoremstyle{definition}
\newtheorem{theorem}{Theorem}
\newtheorem{corollary}{Corollary}
\newtheorem{definition}{Definition}
\newtheorem{example}{Example}
\newtheorem{proposition}{Proposition}
\author{Chris Camano: ccamano@sfsu.edu}
\title{MATH 370  Lecture 8 }
\date

\newcommand{\Z}{\mathbb{Z}}
\newcommand{\N}{\mathbb{N}}
\newcommand{\Q}{\mathbb{Q}}
\newcommand{\R}{\mathbb{R}}
\newcommand{\C}{\mathbb{C}}
\newcommand{\lcm}{\mathrm{lcm}}
\setlength{\parskip}{0cm}
\renewcommand{\thesubsection}{\arabic{subsection}}
\renewcommand{\thesubsubsection}{\arabic{subsection}.\arabic{subsubsection}}
\bibliographystyle{amsplain}

\begin{document}
\maketitle
\newcommand{\nlim}{\lim_{n\rightarrow\infty}}
\sect{Geometric Series}\\
\[
  \sum_{k=0}^n aq^k=\frac{a(1-q^{n+1})}{1-q}=\frac{a}{1-q}
\]
Any element in Q can be represented as a finite expression.\\\\
A monotone sequence that is bounded by above converges.
\definition Cauchy Sequence: \\
all Cauchy sequences converge, all cauchy sequences are monotone and bounded:
\[
  \forall \epsilon>0\quad \exists N\forall  n,m>N: |s_n-s_m|<\epsilon
\]
This is saying that for any two points in the sequence past M that the difference of their values when evaluated in the sequence is less tahn epsilon. \\
If this is true the sequence is referred to as being a cauchy sequence. \\\\
If $\nlim$ exits then the sequence is a cauchy sequence.
\begin{proof}
  \[
    \forall \epsilon>0\quad \exists N\quad \forall  n>N: |s_n-L|<\frac{\epsilon}{2}
  \]
  \[
    \forall \epsilon>0\quad \exists N\quad \forall  m>N: |s_n-L|<\frac{\epsilon}{2}
  \]
  \[
    |a_n-a_m|\leq|a_n-L|+|L-a_m|<\frac{\epsilon}{2}+\frac{\epsilon}{2}=\epsilon
  \]
\end{proof}
Being cacuchy and having a limit is equivilant to the completeness of the real numbers. (EXAM QUESTION)\\\\
\definition:
Let $b_m=\{sup\{a_m,a_{m+1},..\}$ then we know that $b_m$ converges since the sequence is decreasing. Thus:
\[
  \nlim b_n=\nlim \sup a_n
\]
$\nlim \sup a_n$ always exists:
Example:
\[
  \nlim \sup \frac{(-1)^n}{n}=0
\]
\[
  \nlim \inf a_n=\nlim \inf \{a_m,a_{m+1},...
\]
\theorem If $\nlim\sup a_n$exists and $\nlim\inf a_n$ and $\lim a_n$ exists then :
\[
  \nlim\sup a_n=\nlim\inf a_n=\nlim a_n
\]
\end{document}
