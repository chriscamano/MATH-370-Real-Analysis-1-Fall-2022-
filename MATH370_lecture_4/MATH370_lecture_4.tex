\documentclass[12pt]{article}
\usepackage[pdftex]{graphicx}
\usepackage{amsmath,amssymb,amsthm}
\usepackage{hyperref}
\pagestyle{empty}
\author{Chris Camano: ccamano@sfsu.edu}
\title{MATH 335  Lecture 2 }
\date

\topmargin -0.6in
\headsep 0.40in
\oddsidemargin 0.0in
\textheight 9.0in
\textwidth 6.5in
\vfuzz2pt
\hfuzz2pt

%%%%Short cuts and formatting%%%%%%%%%%
\newcommand{\q}{\quad}
\newcommand{\tab}{\\\\}
\renewcommand{\labelenumi}{\alph{enumi})}
\newcommand{\sect}[1]{\section*{#1}}

%%%%%%Vector Spaces%%%%%%%%%%%%%%%%%%%
\newcommand{\R}{\mathbb{R}}
\newcommand{\C}{\mathbb{C}}
\newcommand{\F}{\mathbb{F}}
\newcommand{\rtwo}{\mathbb{R}^2}
\newcommand{\mxn}{{m\times n}}

%%%%%%Sets and common phrases%%%%%%%%%
\newcommand{\Axb}{\textbf{Ax=b} }
\newcommand{\Axz}{\textbf{Ax=0} }
\newcommand{\dim}{\text{dim}}
\newcommand{\lc}{linear combination }
\newcommand{\let}{\text{Let }}
\newcommand{\tf}{\therefore}
%%%%%%%%%Analysis%%%%%%%%%%%%%%%%%%%%%
\newcommand{\arr}{\rightarrow}
\newcommand{\xlim}{\lim_{x\rightarrow \infty}}
\newcommand{\Z}{\mathbb{Z}}
\newcommand{\Q}{\mathbb{Q}}
\newcommand{\N}{\mathbb{N}}
\newcommand{\ep}{\varepsilon}
\newcommand{\i}{\text{ if }}
\newcommand{\and}{\text{ and }}
%%%%%% Theorem formatting%%%%%%%%%%%
\newtheorem{thm}{Theorem}[section]
\newtheorem{cor}[thm]{Corollary}
\newtheorem{lem}[thm]{Lemma}
\newtheorem{prop}[thm]{Proposition}
\theoremstyle{definition}
\newtheorem{defn}[thm]{Definition}
\theoremstyle{remark}
\newtheorem{rem}[thm]{Remark}
\numberwithin{equation}{section}
\everymath={\displaystyle}

\begin{document}
\maketitle
\defn Archemedian principle\\
For any step size r you can traverse the real number line from a to b with this step and pass b at some point. This principle is later used to impy the denisty over the rationals in the sapce of real numbers. \\\\
\defn Density Property;
\[
  \forall a,b \in \R, \exists  , r \in \mathbb{Q} : a<r<b
\]

Approximating square roots:
Let :
\[
  S=\{r: r^2<2\}
\]
S does not have a maximum due to the density of the reals. \\\\
\thm Largrange's theorem: The only irrational numbers that can be expressed as a period sequence of recurring fractions are quadratic numbers. \\\\

Can you make a theory about cubic root expansion?

Homework: \\\\
Pick :
\[
  0<r\quad r^2<2, r\in \Q
\]
Show that: \[
  w=\frac{4}{r^2+2}r
\]
is irrational and also show $r<w$ and also show that $w^2<2$. Then conclude that the set :
\[
  S=\{r\in \Q: r^2<2\}
\]
Does not have a maximum There does not exist least upper bound in the set of rationals. \\\\also iterate the function and compute values: plot the function as well and observe the behavior. The intersection is at the square root of x. \\\\
compute the derivative of the fucntion at the fixed point 
\end{document}
