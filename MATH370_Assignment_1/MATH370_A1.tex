
\documentclass[12pt]{article}
\usepackage[pdftex]{graphicx}
\usepackage{amsmath,amssymb,amsthm}
\usepackage{hyperref}
\pagestyle{empty}
\author{Chris Camano: ccamano@sfsu.edu}
\title{MATH 370  Homework 1 }
\date

\topmargin -0.6in
\headsep 0.40in
\oddsidemargin 0.0in
\textheight 9.0in
\textwidth 6.5in
\vfuzz2pt
\hfuzz2pt

%%%%Short cuts and formatting%%%%%%%%%%
\newcommand{\q}{\quad}
\newcommand{\tab}{\\\\}
\renewcommand{\labelenumi}{\alph{enumi})}
\newcommand{\sect}[1]{\section*{#1}}

%%%%%%Vector Spaces%%%%%%%%%%%%%%%%%%%
\newcommand{\R}{\mathbb{R}}
\newcommand{\C}{\mathbb{C}}
\newcommand{\F}{\mathbb{F}}
\newcommand{\Q}{\mathbb{Q}}
\newcommand{\rtwo}{\mathbb{R}^2}
\newcommand{\mxn}{{m\times n}}

%%%%%%Sets and common phrases%%%%%%%%%
\newcommand{\Axb}{\textbf{Ax=b} }
\newcommand{\Axz}{\textbf{Ax=0} }
\newcommand{\dim}{\text{dim}}
\newcommand{\lc}{linear combination }
\newcommand{\let}{\text{Let }}
\newcommand{\tf}{\therefore}
%%%%%%%%%Analysis%%%%%%%%%%%%%%%%%%%%%
\newcommand{\arr}{\rightarrow}
\newcommand{\xlim}{\lim_{x\rightarrow \infty}}
\newcommand{\Z}{\mathbb{Z}}
\newcommand{\N}{\mathbb{N}}
\newcommand{\ep}{\varepsilon}
\newcommand{\i}{\text{ if }}
\newcommand{\and}{\text{ and }}
%%%%%% Theorem formatting%%%%%%%%%%%
\newtheorem{thm}{Theorem}[section]
\newtheorem{cor}[thm]{Corollary}
\newtheorem{lem}[thm]{Lemma}
\newtheorem{prop}[thm]{Proposition}
\theoremstyle{definition}
\newtheorem{defn}[thm]{Definition}
\theoremstyle{remark}
\newtheorem{rem}[thm]{Remark}
\numberwithin{equation}{section}
\everymath={\displaystyle}


\begin{document}
\maketitle


\[
  \alpha
\]




\sect{Problem 1}
\begin{proof}Show that:
  \[
    \sqrt{3}-\sqrt{2}\notin \Q
  \]
  Let S=$\sqrt{3}-\sqrt{2}$
\begin{align*}
  &S^2=3-2\sqrt{3}\sqrt{2}+2\\\\
  &\frac{5-S^2}{2}=\sqrt{6}
\end{align*}

\begin{proof}
  Suppose that $\left(\frac{p}{q}\right)^2=6\quad p,q\in \Z$ and gcd(p,q)=1 (realtively prime).\\
  this implies:
  \[
    p^2=6q^2
  \]
  If $a|bc$ then $a|b$ or $a|c$. This implies that either $6|p$ or $6|p$ meaning $6|p$.\\
  If $6|p$ this implies $p=6k,k\in \N$.\\
  Returning to the original problem we now have:
  \[
    6k^2=q^2
  \]
  Which implies that q is divisible by 6 this contradicts the original statement that the rational number $\frac{p}{q}$ is realtively prime.\\\\

\end{proof}
Therefore $\sqrt{6}$ is not rational and inturn $\frac{5-S^2}{2}$ is not rational by equivilancy.
\end{proof}

\sect{Problem 2} $1^2+2^2+..n^2=\frac{1}{6}n(n+1)(2n+1)}$

\begin{proof}
  Suppose $$1^2+2^2+..n^2=\frac{1}{6}n(n+1)(2n+1)}$$ is true: \\
  Base Case:n=2:\\\\
  $$1+4=\frac{1}{6}(2)(3)(5)$$
  $$5=5$$.
  P(k):\\\\
  $$\sum_{i=1}^{k}i^2=\frac{(k)(k+1)(2k+1)}{6}$$
  P(k+1):\\\\
  $$\sum_{i=1}^{k+1}i^2=\frac{(k+1)(k+2)(2(k+1)+1)}{6}$$\\\\
  $$\sum_{i=1}^ki^2+(k+1)^2=\frac{(k+1)(k+2)(2k+3)}{6}$$
  \[
    \frac{k(k+1)(2k+1)}{6}+(k+1)^2=\frac{(k+1)(k+2)(2k+3)}{6}
  \]
  \[
    \frac{2k^3+9k^2+13k+6}{6}=\frac{2k^3+9k^2+13k+6}{6}
  \]
\end{proof}
\sect{Problem 3}
Use induction to show"
\[
  \sum_{k=1}^nk^4=\frac{n^5}{5}+\frac{n^4}{2}+\frac{n^3}{3}-\frac{n}{30}
\]
\begin{proof}
  Base Case: n=1:
  \[
    1^4=\frac{1^5}{5}+\frac{1^4}{2}+\frac{1^3}{3}-\frac{1}{30}
  \]
  \[
    1=1
  \]
  P(k):  $$\sum_{i=1}^ki^4=\frac{k^5}{5}+\frac{k^4}{2}+\frac{k^3}{3}-\frac{k}{30}$$
  P(k+1):
  \[
    $$\sum_{i=1}^{k+1}i^4=\frac{(k+1)^5}{5}+\frac{(k+1)^4}{2}+\frac{(k+1)^3}{3}-\frac{k+1}{30}$$
  \]
  \begin{align*}
    &\sum_{i=1}^{k+1}i^4=\frac{(k+1)^5}{5}+\frac{(k+1)^4}{2}+\frac{(k+1)^3}{3}-\frac{k+1}{30}\\\\
    &\sum_{i=1}^{k}i^4+(k+1)^4=\frac{(k+1)^5}{5}+\frac{(k+1)^4}{2}+\frac{(k+1)^3}{3}-\frac{k+1}{30}\\\\
    &\frac{k^5}{5}+\frac{k^4}{2}+\frac{k^3}{3}-\frac{k}{30}+(k+1)^4=\frac{(k+1)^5}{5}+\frac{(k+1)^4}{2}+\frac{(k+1)^3}{3}-\frac{k+1}{30}\\\\
    &\frac{6k^5+15k^4+10k^3-k}{30}+(k+1)^4=\frac{6k^5+45k^4+130k^3+180k^2+119k+30}{30}\\\\
    &\frac{6k^5+45k^4+130k^3+180k^2+119k+30}{30}=\frac{6k^5+45k^4+130k^3+180k^2+119k+30}{30}\\\\
  \end{align*}
\end{proof}
\sect{Problem 4}
\begin{proof}
  Show that: $\sqrt{2+\sqrt{2}}$. \\\\
  Suppose that  $\sqrt{2+\sqrt{2}}=\frac{p}{q}$ where p and q are relatively prime\\

  \begin{align*}
    &2+\sqrt{2}=\frac{p^2}{q^2}\quad p,q\in \Z\\\\
    &\sqrt{2}=\frac{p^2-2q^2}{q^2}\\\\
    &2=\frac{(p^2-2q^2)^2}{q^4}\\\\
    &2q^4=(p^2-2q^2)^2\\\\
    &2q^4=p^4-4p^2q^2+4q^4\\\\
    &p^4=-4p^2q^2+4q^4-2q^4\\\\
    &p^4=2(-2p^2q^2+2q^4-q^4)\\\\
    &\therefore 2|p\rightarrow p=2k, k \in \Z\\\\
    &2=\frac{((2k)^2-2q^2)^2}{q^4}\\\\
    &2=\frac{(4k^2-2q^2)^2}{q^4}\\\\
    &2q^4=(4k^2-2q^2)^2\\\\
    &2q^4=16k^4-16k^2q^2+4q^2\\\\
    &q^4=8k^4-8k^2q^2+2q^2\\\\
    &q^4=2(4k^4-4k^2q^2+q^2)\\\\
    &\therefore 2|q\\\\
  \end{align*}
  $2|p$and $2|q$ which contradicts the notion that p and q are relatively prime.
\end{proof}
\sect{Problem 5}
\begin{proof}
  Find an nsuch that $n^{10} <2^n$. Then use induction to show that the inequality remains
true for all numbers greater or equal to the one that you found, $n>1$\\\\
Base case: n =59 : \\\\
\[
  59^{10}<2^{59}
\]
\[
   5.1111675e+17<5.7646075e+17
\]
P(k):\\
\[
  k^{10}<2^k
\]

P(k+1):
\begin{align*}
  &k^{10}<2^k\\\\
  &2k^{10}<2^{k+1}\\\\
  &k^{10}+k^{10}<2^{k+1}
\end{align*}
$$
  k^{10} +\sum_{i=1}^9\binom{10}{i}k^{10-i}+k(k^9-\sum_{i=1}^9\binom{10}{i}k^{10-i-1})<2^{k+1}
$$
the final term of $(k+1)^{10}$ is
\[
  \binom{10}{10}x^{10-10}=1
\]
For all $k\geq 59$ $$ k(k^9-\sum_{i=1}^9\binom{10}{i}k^{10-i-1})>1$$ so
\begin{align*}
  &k^{10} +\sum_{i=1}^9\binom{10}{i}k^{10-i}+1<k^{10} +\sum_{i=1}^9\binom{10}{i}k^{10-i}+k(k^9-\sum_{i=1}^9\binom{10}{i}k^{10-i-1})<2^{k+1}\\\\
    &(k+1)^{10}<x^{10} +\sum_{i=1}^9\binom{10}{i}k^{10-i}+k(k^9-\sum_{i=1}^9\binom{10}{i}k^{10-i-1})<2^{k+1}
\end{align*}
\[
  (k+1)^{10}<2^{k+1}
\]
\end{proof}
\end{document}
