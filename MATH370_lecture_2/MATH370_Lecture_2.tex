\documentclass[12pt]{article}
\usepackage[pdftex]{graphicx}
\usepackage{amsmath,amssymb,amsthm}
\usepackage{hyperref}
\pagestyle{empty}
\author{Chris Camano: ccamano@sfsu.edu}
\title{MATH 370  Lecture 2 }
\date

\topmargin -0.6in
\headsep 0.40in
\oddsidemargin 0.0in
\textheight 9.0in
\textwidth 6.5in
\vfuzz2pt
\hfuzz2pt

%%%%Short cuts and formatting%%%%%%%%%%
\newcommand{\q}{\quad}
\newcommand{\tab}{\\\\}
\renewcommand{\labelenumi}{\alph{enumi})}
\newcommand{\sect}[1]{\section*{#1}}

%%%%%%Vector Spaces%%%%%%%%%%%%%%%%%%%
\newcommand{\R}{\mathbb{R}}
\newcommand{\C}{\mathbb{C}}
\newcommand{\F}{\mathbb{F}}
\newcommand{\rtwo}{\mathbb{R}^2}
\newcommand{\mxn}{{m\times n}}

%%%%%%Sets and common phrases%%%%%%%%%
\newcommand{\Axb}{\textbf{Ax=b} }
\newcommand{\Axz}{\textbf{Ax=0} }
\newcommand{\dim}{\text{dim}}
\newcommand{\lc}{linear combination }
\newcommand{\let}{\text{Let }}
\newcommand{\tf}{\therefore}
%%%%%%%%%Analysis%%%%%%%%%%%%%%%%%%%%%
\newcommand{\arr}{\rightarrow}
\newcommand{\xlim}{\lim_{x\rightarrow \infty}}
\newcommand{\Z}{\mathbb{Z}}
\newcommand{\N}{\mathbb{N}}
\newcommand{\ep}{\varepsilon}
\newcommand{\i}{\text{ if }}
\newcommand{\and}{\text{ and }}
%%%%%% Theorem formatting%%%%%%%%%%%
\newtheorem{thm}{Theorem}[section]
\newtheorem{cor}[thm]{Corollary}
\newtheorem{lem}[thm]{Lemma}
\newtheorem{prop}[thm]{Proposition}
\theoremstyle{definition}
\newtheorem{defn}[thm]{Definition}
\theoremstyle{remark}
\newtheorem{rem}[thm]{Remark}
\numberwithin{equation}{section}
\everymath={\displaystyle}


\begin{document}
\maketitle
\sect{Opening Comments}\\
Homework information: \\
In the last lecture we concluded that there exist small gaps as you zoom into the real number line between rational numbers due to the existence of irrational numbers. \\
\defn Dense set\\
The set of rational numbers is a dense set, meaning that between any two numbers there exists a third number
\\
An example of a dense proper subset of the rational numbers would be;
\[
   B=\left\{\frac{k}{2^n}: k \in \Z, n \in \N\right\}
\]
as the value of n increments across the natural numbers we are partitioning the pre existing subset into smaller halves. This set has the property that:
\[
  \forall a\in B, b\in B, \exists c: a<c<b
\]\\
\begin{proof}
  Suppose that $\left(\frac{p}{q}\right)^2=3\quad p,q\in \Z$ and gcd(p,q)=1 (realtively prime).\\
  this implies:
  \[
    p^2=3q^2
  \]
  If $a|bc$ then $a|b$ or $a|c$. This implies that either $3|p$ or $3|p$ meaning $3|p$.\\
  If $3|p$ this implies$p=3k,k\in \N$.\\
  Returning to the original problem we now have:
  \[
    3k^2=q^2
  \]
  Which implies that q is divisible by 3 this contradicts the original statement that the rational number$\frac{p}{q}$ was written in reduced form. 
\end{proof}
\end{document}
