\documentclass[11pt]{article}
\usepackage{amssymb,latexsym,amsmath,amsthm,graphicx, cite}
 \author{Chris Camano: ccamano@sfsu.edu}
 \title{MATH 370  Homework 2 }
 \date

\usepackage{mathptmx}
\usepackage{multirow}
\usepackage{float}
\restylefloat{table}
\hoffset=0in
\voffset=-.3in
\oddsidemargin=0in
\evensidemargin=0in
\topmargin=0in
\textwidth=6.5in
\textheight=8.8in
\marginparwidth 0pt
\marginparsep 10pt
\headsep 10pt

\theoremstyle{definition}  % Heading is bold, text is roman
\newtheorem{theorem}{Theorem}
\newtheorem{corollary}{Corollary}
\newtheorem{definition}{Definition}
\newtheorem{example}{Example}
\newtheorem{proposition}{Proposition}



\newcommand{\Z}{\mathbb{Z}}
\newcommand{\N}{\mathbb{N}}
\newcommand{\Q}{\mathbb{Q}}
\newcommand{\R}{\mathbb{R}}
\newcommand{\C}{\mathbb{C}}

\newcommand{\lcm}{\mathrm{lcm}}



\setlength{\parskip}{0cm}
%\renewcommand{\thesection}{\Alph{section}}
\renewcommand{\thesubsection}{\arabic{subsection}}
\renewcommand{\thesubsubsection}{\arabic{subsection}.\arabic{subsubsection}}
\bibliographystyle{amsplain}

%\input{../header}


\begin{document}

\maketitle
For problem one please see attached pdf image. This image should display prior to the work done in Latex. If for any reason this image is not included in this assignment please email me at the address above and I will send an additional copy.

\sect{Problem 2}
Write down all conditions of an ordered field. State the condition that an
ordered field be dense. Google an example of an ordered field that is not
dense.[The extent to which you wish to understand what you google is up
to you and your time.]
\\\\
\textbf{Properties of a field}:
\begin{enumerate}
\item Associativity of addition and multiplication: a + (b + c) = (a + b) + c, and a ⋅ (b ⋅ c) = (a ⋅ b) ⋅ c.
\item Commutativity of addition and multiplication: a + b = b + a, and a ⋅ b = b ⋅ a.
\item Additive and multiplicative identity: there exist two different elements 0 and 1 in F such that a + 0 = a and a ⋅ 1 = a.
\item Additive inverses: for every a in F, there exists an element in F, denoted −a, called the additive inverse of a, such that a + (−a) = 0.
\item Multiplicative inverses: for every a $\neq$ 0 in F, there exists an element in F, denoted by a-1 or 1/a, called the multiplicative inverse of a, such that a ⋅ a−1 = 1.
\item Distributivity of multiplication over addition: a ⋅ (b + c) = (a ⋅ b) + (a ⋅ c).
\end{enumerate}
\textbf{Additional Properties of an Ordered Field:}
An order structure $\leq$  satisfying :
\begin{enumerate}
  \item Given a and b either $a \leq b$ or $b\leq a$
  \item If $a\leq b $ and $b \leq a $ then a=b
  \item If $a\leq b$ and $b \leq c$ then $a\leq c$
  \item If $a \leq b$ then $a+c\leq b+c$
  \item If $a \leq b$ and $0 \leq c$ then $ac \leq bc$
\end{enumerate}
These are the five properties listed in the Ross textbook, however I suspect that a more general description exists for an ordered field. Some research revealed to me that an ordered field is a pair consisting of a field $\mathbb{F}$and a subset of that field $,\mathbb{F}^+$ satisfying:
\begin{enumerate }
  \item $\forall a, b \in \mathbb{F}^+ , a+b \in \mathbb{F}$
  \item $\forall a,b \in \mathbb{F}^+, ab\in \mathbb{F}^+$
  \item $\forall a \in F$ either :
  \[
    a\in \mathbb{F}^+ \quad -a \in \mathbb{F}^+ \quad a=0
  \]
  is true
\end{enumerate }
When doing research on complex eigenvalues over the summer, I learned that the complex numbers form an unordered field as I was not able to design an algorithm that sorted them in a meanginful way besides magnitude which did not account for conjugate pairs. \\\\
After researching online I was not able to find an example of an ordered field that was not dense, there were some interesting debates about the categorical classification of ordered fields on mathstack exchange but I could not find a single psuedo credible source describing an ordered field that is not dense. This leads me to believe that all ordered fields are dense for the following reasons: \\
One a finite field cannot be dense thus an ordered field must at the very least be an infinite set. If this set was infinite but not dense then it would consist of integer values or some similar notion of a single multiple of a number in a number system. If this were the case then the axioms of an ordered field would not hold becuase ...


p(x)/q(x) where p and q are polynomials where you take max of the degree over the polynomials archimidean field and density is equivilant




\sect{Problem 3. }\\
Let {L, U} be a partition of rationals $\Q$ defined by $L = \{l \in \Q|l^2 < 2\}\cup \Q^-$
and $U = \{u \in \Q|u^
2 > 2\} \cap \Q^+$
\begin{enumerate}

  \item (a) Let
  \[
    f(l)=\frac{4}{l^2+2}|l|
  \] Show that for every $l \in  L, l < f(l)$ and f(l) $\in$ L.

\begin{proof}
  for all $l \in L$ the term : \
  \[
    q=\frac{4}{l^2+2}>1
  \]
  Since in order for q to be equal to one or less l would have to be a solution to $l^2+2$. By construction of L this implies q is always greater than one acting as a positive scalar on the absolute value of l guarenteeing that $f(l)>l$\\\\
  To show that $f(l)\in L$ consider the following: Suppose that $f(l)\notin \L$ this implies that $f(l)^2 \geq 2$ or equivilantly:
  \[
    \left(\frac{4}{l^2+2}|l|\right)^2\geq 2
  \]
  \begin{align*}
    &\frac{16l^2}{(l^2+2)^2}\geq 2\\
    &16l^2\geq 2(l^2+2)^2\\
    &8l^2\geq l^4+2l^2+4\\
    &0\geq l^4-4l^2+4\\
    &0\geq (l-2)^2
  \end{align*}

  This implies that f(l)^2$\geq 2$ when $(l^2-2)^2\leq 0$ Solving for 0 the roots of this polynomial are $\pm\sqrt{2}$ but by construction $\sqrt{2} \notin L$ therefore we are left with  $(l^2-2)^2>0 $ for all $l\in L$ we now need to show that this cannot be true. Consider the case when we choose an arbitrary $l \in L$ we then have some value $l^2$ less than 2 minus 2 meaning that this difference is always negative. Since we are then squaring a negative number for all l in L $(l^2-2)^2>0$ this implies that $(l^2-2)^2$ is never less than or equal to zero and in turn that it cannot be the case that\[
    f(l)^2\geq 2
  \] due to the fact that this is a re-expression of the same inequality.  from this we conclude  that $f(l)^2<2$ and thus $f(l)\in L$


\end{proof}
  \item (b) Conclude that L does not have a largest element.\\
  $L\subset \mathbb{Q}$ therefore by the densness of $\Q$ extends to L as well which implies there does not exist a largest number in the set L. Another approach would be to consider the point halfway between $\sqrt{2}$ and a proposed maximum of L. Since densness gives us the property that forall a and b there exists c such that: $a<c<b$ this implies there exists a slightly larger number then a proposed maximum a.
  \item (c)
  . Show that for every $u \in  U, g(u) < u$ and g(u) $\in$U\\
  This proof will follow a similar construction to that of part a. Let us first define the function as follows:\\
  \[
    g(u)=\frac{u+\frac{2}{u}}{2}=\frac{u^2+2}{2u}=\frac{u}{2}+\frac{1}{u}
  \]
  by partial fraction decomposition. \\
  Note that this is equivilant to dividing u in half and then adding a number less than one. In order for g(u) to be $\geq $ u then this would mean solving the expression: \[
    g(u)=u
  \]
  \begin{align*}
  &  \frac{u^2+2}{2u}=u\\
  & u^2+2=2u^2\\
  &2=u^2\\
  \end{align*}
  The set U consists of the elements in $\mathbb{Q}$ such that for each$ u\in U$ $u^2>2$  which means that $g(u)$ can then concequently never be equal to u. To show that g(u) cannot be greater than u consider the following. If g(u) was greater than u this would mean that :
  \[
    0<u^2<2
  \]
  however by construction the set U excludes these values ( actually the first half of L here) so g(u) can never be greater than u either meaning we have proven that $\forall u\in U g(u)<u$. To show that $g(u)\in U$ we can recylce quite a bit of logic from the previous proof, if g(u) is always less than u , and $u^2$ is always greater than 2 then g(u) obeys the following inequality:
  \[
    2<g(u)^2<\infty
  \]
  The reason that U is not bounded above is because the intersection of $\{u \in \Q|u^2 > 2\}$ and  $\Q^+$ is the set of all rational numbers such that for a given number its square is larger than two.g(u) goes to infinity as the value of u grows and by the denseness of the rationals we can always find larger u to pick since we are given the intersection with all positive rationals.
  \item (d) Conclude that U does not have a least element.
  Due to the fact that $U \subset \Q $ we can extend the properties of denseness to U as well. This implies that for any contendor point in U claimed to be the smallest element we can find another smaller element by comparing the infimum of the set and the prospective rational and picking a point half way between them such that if c is the contendor:
  \[
    2< \frac{c^2+2}{2}<c^2
  \]
  \item (e) Conclude that U is the set of all rational upper bounds of L.\\
  Since U is the set of all positive rational numbers (thanks to the intersection with positive Q ) with a sqaure greater than two and the supremum of L is the square root of two then we can say confidently that U is the set of all upper bounds for L since for any $u \in U$ and any $l \in L $ $l< u $ as $l^2<2$ and $u^2>2$
  \item (f) Conclude that sup L$\notin \Q$.\\
  The supremum of L is the irrational number $\sqrt{2}$ the set L is constructed such that all values are less than the sqaure root of two. This means that the supremum of L is $\sqrt{2}$ and since $\sqrt{2}$ is irrational $\sup L \notin \Q$
  \item (g) List computed decimal values of rationals f(1), f(f(1)), $f^3$(1) andg(3), g(g(3)), $g^3$(3).
  Here are the values:
  \begin{align*}
    &f(1)= \frac{4}{3}=1.333\\
    &f(f(1))=\frac{24}{17}=1.41176\\
    &f(f(f(1)))=\frac{816}{577}=1.414211\\
    &g(3)=1.833\\
    &g(g(3))=1.46212\\
    &g(g(g(3)))=\frac{72097}{50952}=1.415
  \end{align*}

\end{enumerate}
\sect{Problem 4}\\
Squeezing the most irrational number by continued fractions. Let $g=\frac{\sqrt{5}-1}{2}. Let f(x)=\frac{1}{1+x}$L is the set of odd iteratios of f(1) U is the set of even iterations of f(1).\\
\begin{enumerate}
  \item (a) Write down two elements of L and two elements of U as continued
  fractions and then compute their values and place them on a real
  axis together with g.\\
  Elements of L:
  \[
    f^3(1)=\cfrac{1}{1+\cfrac{1}{1+\cfrac{1}{2}}}=\frac{3}{5}
  \]
  \[
    f^5(1)=\cfrac{1}{1+\cfrac{1}{1+\cfrac{1}{1+\cfrac{1}{1+\cfrac{1}{2}}}}}=\frac{8}{13}
  \]
  Elements of U:\[
    f^2(1)=\cfrac{1}{1+\frac{1}{2}}=\frac{2}{3}
  \]
  \[
    f^4(1)=\cfrac{1}{1+\cfrac{1}{1+\cfrac{1}{1+\frac{1}{2}}}}=\frac{5}{8}
  \]
  I do not have a tikz environment configured so please reference this inequality as evidence that I understand how these are ordered on a number line.

  \[
    f^1<f^3<f^5<f^4<f^2
  \]
  \item (b) Show that odd iterates of 1 form an increasing sequence bounded
  above by g. Show that even iterates of 1 form a decreasingbounded below by g.\\
  \[
    L=\{.5,.6,.615385,.617647,.617978,...\}
  \]
  \[
    U=\{.66,.625,.619048,.618182,.618056
  \]
  \\\\
  We would like to prove that for the odd compositions of f we obtain
  \[
    \mathfrak{f}_{k-1}<\mathfrak{f}_k
  \]
where $\mathfrak{f}$ is the value of the sequce after k compositions, k is an odd number.
  this is to say that:
  \[
    f_{k-1}\circ f_{k-2}\circ \cdots \circ f_1(1)<f_{k}\circ f_{k-1}\circ \cdots \circ f_1(1)
  \]
Suppose for the sake of contradiction that there exist some odd number of composures k such that:
\[
  \mathfrak{f}_{k+1}\leq\mathfrak{f}_k<g
\]
then this is the same as saying:
\[
    f_{k+1}\circ f_{k}\circ \cdots \circ f_1(1)\leq f_{k}\circ f_{k-1}\circ \cdots \circ f_1(1)<g
\]
We now compose both side of the inequality of the corresponsing inverse $\mathfrak{f}^{-1}_{2,..,k}$ as follows:
\[
  \left[ f_{k+1}\circ f_{k}\circ \cdots \circ f_1(1)\right]\circ \left[f_2^{-1}\circ \cdots f_k^{-1}\right] \leq \left[ f_{k}\circ f_{k-1}\circ \cdots \circ f_1(1)\right]\circ \left[f_2^{-1}\circ \cdots f_k^{-1}\right]<g
\]
which after composition leaves behind:
\begin{align*}
  &  f\circ f(1)\leq f(1)<g\\
  &   f(\frac{1}{2})\leq \frac{1}{2}<g\\
  &\frac{1}{\frac{3}{2}}\leq \frac{1}{2}<g\\
  &\frac{2}{3}\leq \frac{1}{2}<g
\end{align*}
So we arrive at a contradiction which implies that instead.\[
   \mathfrak{f}_k< \mathfrak{f}_{k+1}
\]
which allows us to conclude that L is an increasing set. note that this also gives us $ \mathfrak{f}_k< \mathfrak{f}_{k+1}<g$ when correcting the erroneous inequality. The proof for even compositions is symmetric in nature.
  \item (c) Conclude that every element of U is an upper bound of L.
  Since every element of U is greater than the supremum of L every $u\in U$ acts as an upper bound for L.


  \item (d) Conclude that sup L = inf U = g.\\
  L is bounded above and S has a least upper bound, that being the value g. By definition this means
  $\forall l\in L, l <g$ thus supL = g.
  Likewise U is bounded below and has a greatest lower bound also g. By definition this means
  $\forall u\in U, u>g$ thus infU=g. \\
  We then arrive at the implied relation:
  \[
    supL=infU.
  \]

\end{enumerate}


\end{document}
