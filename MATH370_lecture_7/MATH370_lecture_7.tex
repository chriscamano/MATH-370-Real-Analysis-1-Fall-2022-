\documentclass[11pt]{article}
\usepackage{amssymb,latexsym,amsmath,amsthm,graphicx, cite}
 \author{Chris Camano: ccamano@sfsu.edu}
 \title{MATH 370  lecture 7 }
 \date

\usepackage{mathptmx}
\usepackage{multirow}
\usepackage{float}
\restylefloat{table}
\hoffset=0in
\voffset=-.3in
\oddsidemargin=0in
\evensidemargin=0in
\topmargin=0in
\textwidth=6.5in
\textheight=8.8in
\marginparwidth 0pt
\marginparsep 10pt
\headsep 10pt

\theoremstyle{definition}  % Heading is bold, text is roman
\newtheorem{theorem}{Theorem}
\newtheorem{corollary}{Corollary}
\newtheorem{definition}{Definition}
\newtheorem{example}{Example}
\newtheorem{proposition}{Proposition}



\newcommand{\Z}{\mathbb{Z}}
\newcommand{\N}{\mathbb{N}}
\newcommand{\Q}{\mathbb{Q}}
\newcommand{\R}{\mathbb{R}}
\newcommand{\C}{\mathbb{C}}

\newcommand{\lcm}{\mathrm{lcm}}



\setlength{\parskip}{0cm}
%\renewcommand{\thesection}{\Alph{section}}
\renewcommand{\thesubsection}{\arabic{subsection}}
\renewcommand{\thesubsubsection}{\arabic{subsection}.\arabic{subsubsection}}
\bibliographystyle{amsplain}

%\input{../header}
\newcommand{\xlim}{\lim_{x\rightarrow\infty}}

\begin{document}
\maketitle
  \sect{Formalization of Zeno's paradox:} \\
  \[
    a_n=1-\frac{1}{2^n}
  \]
A random fact:
\[
  \xlim \sum_{i=1}^n\frac{1}{k}=\infty
\]
For all series of the form:
\[
  \xlim \sum_{i=1}^n\frac{1}{i^p}
\]
The limit is $\infty$ unless p is one. \\\\

$$a_n=(1+\frac{1}{n})^n$$
\newcommand{\nlim}{\lim_{n\rightarrow \infty}}
\[
  \nlim (1+\frac{1}{n})^n=e
\]\\\\

\theorem
A monotone sequence that is bounded converges: Assume we have an increasing sequence . Since we have a bound on the sequence denoted as M we then have:
\[
  \exists_M \forall_n a_n< M
\]
We want to show that the limit of $a_n$ exists, in order to do this we consider the nature of what it means to be bounded. The fact that the set it bounded implies the existince of a supremum for the sequence. This gives the following. \\
Due to the completness of real numbers there exists some number S such that S acts as the supremum for the sequence. S is then by definition of supremum the least upper bound for all elements in the sequence. \\
We show that the limit of the sequence as $n\rightarrow \infty$ is equal to S. To show this we need to show that:
\[
  \forall_{\epsilon >0}\exists_N:\forall_{n>N}|a_n-S|<\epsilon
\]
\begin{align*}
  &s-a_n<\epsilon\\
  &s<\epsilon+a_n\\
\end{align*}
$s-\epsilon$ is not an
upper bound for $a_n$ since S is the supremum which implies there exists N such that $a_N> S-\epsilon$ so:
\[
  \forall_{n>N}a_n\geq a_N >s-\epsilon
\]
Since the sequence is increasing. so we have proven that $a_n\geq s-\epsilon$\\\\
Other theorems from the textbook:
\\
Given $s_n,t_n$ where $\nlim s_n=s,\nlim t_n=t$ then
\[
  \nlim s_nt_n=st
\]
Additional Theorem;
\[
  (s_n\rightarrow \infty) \rightarrow \frac{1}{s_n}\rightarrow \infty
\]



















\end{document}
